\documentclass[12pt, a4paper]{report}

\usepackage{fyp}


%%these packages are not really necessary if you dont need the code and proofs environments
%%so if you like you can delete from here till the next comment
%%note that there are some examples below which obviously won't work once you remove this part
\usepackage{verbatim}
\usepackage{amsfonts}
\usepackage{amsmath}
\usepackage{amssymb}
\usepackage{amsthm}
\usepackage{lmodern}
\usepackage{geometry}
\usepackage[scaled=0.85]{beramono}

%%this environment is useful if you have code snippets
\newenvironment{code}
{\footnotesize\verbatim}{\endverbatim\normalfont}

%%the following environments are useful to present proofs in your thesis
\theoremstyle{definition}
\newtheorem{definition}{Definition}[section]
\theoremstyle{definition}%plain}
\newtheorem{example}{Example}[section]
\theoremstyle{definition}%remark}
\newtheorem{proposition}{Proposition}[section]
\theoremstyle{definition}%remark}
\newtheorem{lemma}{Lemma}[section]
\theoremstyle{definition}%remark}
\newtheorem{corollary}{Corollary}[section]
\theoremstyle{definition}%remark}
\newtheorem{theorem}{Theorem}[section]
%%you can delete till here if you dont need the code and proofs environments



\setlength{\headheight}{15pt}
%\overfullrule=15pt


\begin{document}



%%make sure to enter this information
\title{Program Analysis: Towards the Analysis of CPython Bytecode
    \large{Progress Report}
}
\author{André Theuma}
\date{17/12/2021}
\supervisor{Dr. Neville Grech}
\department{Faculty of ICT}
\universitycrestpath{uom-logo}
\submitdate{17/12/2021} 

\frontmatterprogreport

\begin{abstract}
    \par Software systems are heavily relied on by businesses for their day to day management and are also used to protect critical data.
    In this day and age having a secure system is more important than ever before, thus one must ensure that the system's security is tested out and analysed beforehand.
    \par Program Analysis is a process by which a system's correctness may be tested. In this dissertation, a Static Analysis Framework shall be implemented for its use in the
    CPython environment.

\end{abstract}

\tableofcontents

\newgeometry{top=2cm}
\listoffigures
\restoregeometry

%%\listoftables



\mainmatter

\chapter{Introduction \& Motivation}
    \par Program analysis is the (now automated) process by which a program can be analysed %\cite{nielson2004principles}.
    The results produced 
    by the analysis can then be used for one of two things; Program Optimization or Program Correctness. There are two different
    types of analyses, Static analysis and Dynamic analysis. Typically both analyses are used in conjunction with each other, but in different
    production phases. 
    \par Static analysis is used in the development phase of a program, prior to testing. With this analysis, the root cause 
    of many basic vulnerabilities is found.
    \par Dynamic analysis is performed during runtime and is very resource intensive, in comparison to Static analysis. 
    Dynamic program analysis is more commonly known as correctness runtime debugging. %\cite{bruegge1993framework}

    \par In this dissertation, the main topic of focus shall be on disassembling CPython source code into 
    CPython bytecode and performing a relevant analysis on the produced relations.
    CPython is the most widely used implementation of Python, which is one of the most popular languages. 
    Its use ranges from web-development to enterprise solutions and embedded development. As of today, it is 
    rated as one of the languages used mostly by developers and in several industries %\cite{pyStat}. 

    The relations produced after the disassembly will then be broken down into a more elementary form 
    of relation which is simpler to digest, for the user. Different queries will then be performed on these 
    new relations, depending on the use cases needed, finally producing facts. The query system will be 
    of a Soufflé Datalog implementation so as to facilitate the process.
    \section{Non-Triviality}
    \par Performing Static Program analysis is non-trivial as it forms an essential part to providing secure software and protecting data
    from possible attackers. Program correctness essentially provides data confidentiality and data integrity. %\cite{moller2012static} 

    
\chapter{Background Research \& Literature Review}

    \par The project is split into two main sections. The disassembly of functions into bytecode with it being processed by 
    the bytecode analyser would form part of the first section, and the Soufflé queries would pertain to the second section of 
    this project.

    \section{Bytecode} In CPython we find that the source code is firstly translated into an intermediate language, known as the {\fontfamily{cmtt}\selectfont 
    bytecode%
    },
    and is then passed through and executed by the virtual machine (PVM) %\cite{ike2015inside},
    which is written in C, hence CPython.
    The {\fontfamily{cmtt}\selectfont 
    bytecode%
    } is exposed by the use of the {\fontfamily{cmtt}\selectfont 
    dis%
    }
    module provided by Python %\cite{disModule}.
    The {\fontfamily{cmtt}\selectfont 
    dis%
    } module provides the user
    with the bytecode API wrapper, in which multiple functions may be used to exploit different properties of the source code.
    For bytecode analysis these functions are used from the aforementioned module; 
    {\fontfamily{cmtt}\selectfont 
    dis.Bytecode()%
    } %\cite{disModule} and {\fontfamily{cmtt}\selectfont 
    dis.dis()%
    %\cite{disModule}.
    The former iterates over each bytecode instance as an {\fontfamily{cmtt}\selectfont 
    Instruction%
    } instance %\cite{disModule}. 
    The latter process the source code line by line, and disassembles every object 
    into its corresponding bytecode representation. The bytecode analyser that will be implemented, will disassemble methods
    and classes. When a class is passed to the disassembler, the disassembler disassembles all the methods in the class object,
    printing every bytecode instruction on a new line.
    The {\fontfamily{cmtt}\selectfont 
    Instruction%
    } 
    instance holds the following attributes; 
    {\fontfamily{cmtt}\selectfont 
     opcode%
    },
    {\fontfamily{cmtt}\selectfont 
     opname%
    },
    {\fontfamily{cmtt}\selectfont 
     arg%
    },
    {\fontfamily{cmtt}\selectfont 
     argval%
    },
    {\fontfamily{cmtt}\selectfont 
     argrepr%
    },
    {\fontfamily{cmtt}\selectfont 
     offset%
    },
    {\fontfamily{cmtt}\selectfont 
     starts\_line%
    },
    {\fontfamily{cmtt}\selectfont 
     is\_jump\_target%
    } %\cite{disModule}.

    \par Each attribute contains information which is relevant to the current bytecode instruction.
    
    \section{Soufflé} The second part of the project relies heavily on Soufflé which is an open source
    framework, which performs static program analysis to generate facts %\cite{jordan2016souffle}. 
    Soufflé is extremely fast in comparison
    to SQL/Datalog and has very simple syntax, which makes it a very powerful and user friendly tool. The language is based
    upon Datalog but has state of the art performance %\cite{jordan2016souffle},
    allowing analysis to be performed on large libraries,
    which previously would be infeasible. 
    
    \par Soufflé's fast performance may be credited to its method of evaluating Datalog programs. Soufflé synthesizes and compiles
    the Datalog program into native C++ code. This is possible by firstly compiling the Datalog program into Relational-Algebra-Machine Programs (RAM)
    RAM programs are evaluated by elaborate data indexing techniques and multi-threaded query processing. These RAM programs can then be compiled
    into C++. This is further compiled into machine code, which is what allows for such fast execution. %\cite{varAwareDlog}


\chapter{Methods \& Techniques Planned}
The Methodology planned to carry out this project starts off by creating the CPython bytecode analyser. This analyser will disassemble 
any function or class object passed to it and its bytecode instructions will be inspected and analysed. The PVM (Python Virtual Machine)
stack operations will be taken note of after every bytecode instruction. From this disassembly facts will then be created. The facts produced will
form part of the following spec; a fact might make reference to the value pushed to the stack ({\fontfamily{cmtt}\selectfont 
PushValue%
}), its code object reference ({\fontfamily{cmtt}\selectfont 
Statement\_Code%
}), a reference to the next instruction ({\fontfamily{cmtt}\selectfont 
Statement\_Next%
}), a reference to the instruction's opcode ({\fontfamily{cmtt}\selectfont 
Statement\_Opcode%
}), and a reference to the amount of operations performed on the stack ({\fontfamily{cmtt}\selectfont 
Statement\_Pushes, Statement\_Pops%
}).
\par These facts are to be stored in a Datalog compatible format (csv) so that a Datalog analysis can be performed on them. The type of analysis 
that will be done is still yet to be discussed.


\chapter{Evaluation Strategy}
\par The two main evaluation techniques are the following;
    \begin{itemize}
        \item Evaluate bytecode, analyse each bytecode instruction \& produce facts.
        \item Perform meaningful queries on facts generated by the bytecode analyser.
    \end{itemize}

    From these evaluation techniques, a report might be produced illustrating the queries \& their results.

\chapter{Project Deliverables}

    \begin{itemize}
        \item A Final Year Project Report, including any relevant material used in the project.
        \item The implementation of the CPython Bytecode Analyser, \& the function or class objects used.
        \item The facts generated, the Datalog queries along with their results.
    \end{itemize}

\bibliomatter



\bibliographystyle{abbrv}
    \bibliography{citation}
        
        %\cite{nielson2004principles}
        %\cite{bruegge1993framework}
        %\cite{moller2012static}
        %\cite{pyStat}
        %\cite{aycock1998converting}
        %\cite{ike2015inside}
        %\cite{disModule}
        %\cite{jordan2016souffle}
        %\cite{varAwareDlog}

\end{document}