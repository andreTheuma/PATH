%% For tips on how to write a great abstract, have a look at
%%	-	https://www.cdc.gov/stdconference/2018/How-to-Write-an-Abstract_v4.pdf (presentation, start here)
%%	-	https://users.ece.cmu.edu/~koopman/essays/abstract.html
%%	-	https://search.proquest.com/docview/1417403858
%%  - 	https://www.sciencedirect.com/science/article/pii/S037837821830402X

\begin{abstract}
    \par Program analysis methods offer static compile-time techniques to predict approximations to a set of values or dynamic behaviours which arise during a program's run-time.
    These methods generate useful observations and characteristics about the underlying program, in an automated way.
    \par PATH (Python Analysis Tooling Helper) is a static analysis tool created in this project, which generates a standardized Intermediary Representation for 
    given functions, allowing analysis metrics to be generated from the facts produced by the tool. The goal of this project was to create a framework that generates facts from a function, in addition to an IR that is amenable for further analysis.
    The framework created should simplify the engineering complexity of fact analysis for future use. PATH would disassemble CPython bytecode into a more 
    straightforward representation, making any further possible analyses a simpler task, as analysis can be conduced on the generated IR.
    \par The final findings of the project indicate that performing analysis on the IR generated by PATH is indeed a simpler task than generating facts manually and conducting block analysis without such a framework.
    These results are satisfactory and hold up to the aims of this project.
\end{abstract}