\chapter{CPython VM Reference}
\label{chap:cpythonvmref}
\section{CPython}
		\subsection{Overview}
		\par CPython works transparently, via the \acs{PVM} (visualised in Figure \ref{fig:vm_innards}); an interpreter loop (\textit{ceval.c}) is run and there is no direct translation between the Python code to C \cite[pp.1--2]{aycock1998converting}. 
		The \acs{PVM} is a stack machine whereby \acs{PVM} instructions retrieve their arguments from the stack, just to be placed back onto the stack after execution of the instruction. The Python compiler generates \acs{PVM} code (a \textit{.pyc} file) for the Python \acs{VM} to execute.
		The CPython interpreter resembles a classic interpreter with a straightforward algorithm \cite[pp.2--4]{aycock1998converting}: 
		\begin{itemize}
			\item[1.] Firstly, the opcode of a VM instruction is fetched, along with any necessary arguments.
			\item[2.] Secondly, the instruction is then executed.
			\item[3.] Finally, steps 1-2 are repeated till no more opcodes can be fetched. This is done by raising an exception when an invalid (empty) opcode is found.  
		\end{itemize}

		\begin{figure}[h]
		\centering
			%% flowchart
			\begin{tikzpicture}
				%% setupP
				[
					roundnode/.style={circle, draw=green!60, fill=green!5, very thick, minimum size=20mm},
					rectanglenode/.style={rectangle, draw=red!60, fill=red!40, very thick, minimum size=20mm},
					transparentbox/.style={rectangle,draw=blue!60, fill=blue!0, ultra thick, minimum width = 75mm, minimum height = 40mm},
				]
					
					%% Nodes
					\node [transparentbox] (box) at (1,0) {};
					\node [rectanglenode] (compilernode) at (-1.2,0){Compiler};
					\node [rectanglenode] (vmnode) [right=of compilernode] {Virtual Machine};
					\node [roundnode] (sourcefile) [left=of box] {Source File};
					\node [roundnode] (programnode) [right=of box] {Program Execution};
					\node [rectanglenode] (librarynode) [below=of box] {Library Modules};
					\node [] (text) at (1.02,2.4) {Python Interpreter};
					\node [] (textBytecode) at (0.4,1.4){Bytecode Generation};
					%% Lines
					\draw[->] (librarynode.north) -- (box.south) ;
					\draw[->] (sourcefile.east) -- (compilernode.west);
					\draw[->] (compilernode.east) -- (vmnode.west);
					\draw[->] (vmnode.east) -- (programnode.west);
				\end{tikzpicture}
			
				\caption{Python Code Execution}
				\label{fig:python code execution}
		\end{figure}
		
		\begin{figure}[h]
			\centering    
			\begin{tikzpicture}
				[
					roundnode/.style={circle, draw=green!60, fill=green!5, very thick, minimum size=20mm},
					rectanglenode/.style={rectangle, draw=green!60, fill=green!40, very thick, minimum width=30mm, minimum height = 20mm},
					transparentbox/.style={rectangle,draw=blue!60, fill=blue!0, ultra thick, minimum width = 75mm, minimum height = 40mm},
				]
				
				\node [rectanglenode] (parsetree) {Parse Tree};
				\node [rectanglenode] (ast) [below=of parsetree] at(1.25,-0.5) {Abstract Syntax Tree};
				\node [rectanglenode] (cfg) [below=of ast] at(2.5,-3.1) {Control Flow Graph};
				\node [rectanglenode] (bytecode) [below=of cfg] at(3.75,-5.7) {Bytecode};

				\draw[->] (parsetree.west) -- ++(-1,0)|- (ast.west) ; 
				\draw[->] (ast.west) -- ++(-1,0)|- (cfg.west) ; 
				\draw[->] (cfg.west) -- ++(-1,0)|- (bytecode.west) ; 

			\end{tikzpicture}
			
			\caption{Compiler Innards}
			\label{fig:compiler_innards}
		\end{figure}

		\begin{figure}[H]
			\centering
			\begin{tikzpicture}
				[
					roundnode/.style={circle, draw=green!60, fill=green!5, very thick, minimum size=20mm},
					rectanglenode/.style={rectangle, draw=cyan!60, fill=cyan!20, very thick, minimum width=30mm, minimum height = 20mm},
					transparentbox/.style={rectangle,draw=blue!60, fill=blue!0, ultra thick, minimum width = 75mm, minimum height = 40mm},
				]

				\node [roundnode] (opcode) {NEXTOP()};
				\node [roundnode] (checkargs) [right=of opcode] {NEXTARG()};
				\node [rectanglenode] (switch) [below=of opcode] at(2,-2) {Opcode Case Match};
				\node [] (entrytext) [above=of opcode] {Entry};
				\node [] (exittext) [below=of switch] at(2,-4) {Exit};
				\node [align=center] (label1) [below=of switch] at(-1.3,-4.2) {next opcode is not\\present};
				\node [align=center] (label2) [below=of switch] at(-3.6,-.6) {next opcode\\is present};
				
				\draw[-latex] (entrytext) .. controls +(left:.5cm) and +(left:3cm) .. (opcode.north);
				\draw[-latex] (switch.west) .. controls +(left:1cm) and +(left:3cm) .. (exittext.west);
				\draw[-latex] (switch.west) .. controls +(left:3.2cm) and +(left:2cm) .. (opcode.west);
				\draw[-latex] (checkargs.east) to[in=1, out=2] (switch.east);
				
				\draw[-latex] (opcode.east) -- (checkargs.west);
				
			\end{tikzpicture}

			\caption{VM Innards}
			\label{fig:vm_innards}
		\end{figure}

		This simple algorithm is known as the CPython Evaluation Loop. The evaluation loop is formulated as shown in Listing \ref{lst:ceval_loop}\footnote{Actual code differs from what is presented. The source code has been edited as to be more readable and concise.}.

		\begin{lstlisting}[float=h,language=Python,caption= Evaluation Loop,label=lst:ceval_loop]
			for(**indefinite condition**){
				oparg=null;
				opcode = NEXTOP();
				if(ARGUMENT_PRESENT(opcode)){
					oparg = NEXTARG();
				}
				switch(opcode){
					case **opcode_name**: 
						manipulate stack & set variables accordingly
					...
					...
					...
					default: 
						raise error
				}
			}	
		\end{lstlisting}

		Evaluation is computed frame by frame (See {\bfseries\nameref{subsec:frames}}), with what is essentially a (very long) switch statement; reading every opcode and delegating accordingly.
			
		\subsection{Stacks}
		\label{subsec:stacks}
		\par A stack data structure is a dynamic structure that operates with a LIFO policy \cite[]{intro2009algorithms}. Since CPython does not directly interact with
		the hardware for compilation, it makes both the call stack and stack frames rely on the PVM. In CPython, there is one main stack that the \acs{PVM} requires for proper functionality; the call stack. 
		The other two stacks (\nameref{subsubsec:value_stack} and \nameref{subsubsec:call_stack}) are essential for the proper computation of any variables that there are in the
		frame (See {{\bfseries\nameref{subsec:frames}}). Most instructions manipulate the value stack and the call stack \cite[]{general2018stacks}. The nature of CPython stacks is visualised in Figure \ref{fig:stacks_overview}
			\subsubsection*{Call Stack}
			\label{subsubsec:call_stack}

			\par The call stack contains call-frames (See {\bfseries\nameref{subsec:frames}}). This is the main structure of the running program.
			A function call results in a pushed frame onto the call stack, whilst 
			a return call results in a pop of the function frame off of the stack \cite[]{call2010stack}. A visual representation of a call stack is shown in Figure \ref{fig:call_stack_example}. 
			In this figure, a sample script can be seen run, step by step, showing the frames being pushed onto the call stack and popped from the call stack.
			The first frame pushed onto the frame stack is inevitably called the \lstinline|__main__| call frame.

			\begin{figure}
				\centering

				\begin{lstlisting}[language=Python,caption=Python script, label=lst:callstackexample]
					def foo():
						print("Hello")
					
					def intermediary():
						foo()
						
					def start():
						intermediary()
					
					start()
				\end{lstlisting}

				\begin{tikzpicture}
				[
					roundnode/.style={circle, draw=green!60, fill=green!5, very thick, minimum size=20mm},
					rectanglenode/.style={rectangle, draw=cyan!60, fill=cyan!20, very thick, minimum width=30mm, minimum height = 10mm},
					transparentbox/.style={rectangle,draw=blue!60, fill=blue!0, ultra thick, minimum width = 75mm, minimum height = 40mm},
				]
				
				\node[rectanglenode] (main) at (0,0) {\_\_main\_\_};
				\node [] (text_init) [above=of main] {Initialization};
				
				\node[rectanglenode] (main_1) [right=of main] {\_\_main\_\_};
				\node[rectanglenode] (start) [above=of main_1] at (4.04,-0.5) {start()};
				\node [] (text_start) [above=of start] {\lstinline|start()| pushed};

				\node[rectanglenode] (main_2) [right=of main_1] {\_\_main\_\_};
				\node[rectanglenode] (start_2) [above=of main_2] at (8.085,-0.5) {start()};
				\node[rectanglenode] (intermediary) [above=of start_2] at(8.085,0.5) {intermediary()};
				\node [] (text_int) [above=of intermediary] {\lstinline|intermediary()| pushed};
			
				\node[rectanglenode] (main_3) [right=of main_2] at (9.40,0){\_\_main\_\_};
				\node[rectanglenode] (start_3) [above=of main_3] at (11.92,-0.5) {start()};
				\node[rectanglenode] (intermediary_3) [above=of start_3] at(11.92,0.5) {intermediary()};
				\node[rectanglenode] (foo) [above=of intermediary_3] at(11.92,1.5) {foo()};
				\node [] (text_foo) [above=of foo] {\lstinline|foo()| pushed};

				\end{tikzpicture}
				
				\vspace{20mm}

				\begin{tikzpicture}
					[
						roundnode/.style={circle, draw=green!60, fill=green!5, very thick, minimum size=20mm},
						rectanglenode/.style={rectangle, draw=cyan!60, fill=cyan!20, very thick, minimum width=30mm, minimum height = 10mm},
						transparentbox/.style={rectangle,draw=blue!60, fill=blue!0, ultra thick, minimum width = 75mm, minimum height = 40mm},
					]
					
					\node[rectanglenode] (main) at (8.21,0) {\_\_main\_\_};
					\node [] (text_init) [above=of main] {\lstinline|start()| popped};
					
					\node[rectanglenode] (main_1) [right=of main_2] at(1.86,0) {\_\_main\_\_};
					\node[rectanglenode] (start) [above=of main_1] at (4.38,-0.5) {start()};
					\node [] (text_start) [above=of start] {\lstinline|intermediary()| popped};
		
					\node[rectanglenode] (main_2) [right=of main_3] at (-2.3,0){\_\_main\_\_};
					\node[rectanglenode] (start_2) [above=of main_2] at (0.215,-0.5) {start()};
					\node[rectanglenode] (intermediary) [above=of start_2] at(0.215,0.5) {intermediary()};
					\node [] (text_int) [above=of intermediary] {\lstinline|foo()| popped};
				
					\node[rectanglenode] (main_3) at (-3.6,0){\_\_main\_\_};
					\node[rectanglenode] (start_3) [above=of main_3] at (-3.6,-0.5) {start()};
					\node[rectanglenode] (intermediary_3) [above=of start_3] at(-3.6,0.5) {intermediary()};
					\node[rectanglenode] (foo) [above=of intermediary_3] at(-3.6,1.5) {foo()};
					\node [] (text_foo) [above=of foo] at (-3.6,2.5){\lstinline|foo()| computed};
		
				\end{tikzpicture}

				\caption{Simulation of Call Stack running Listing \ref{lst:callstackexample}}
				\label{fig:call_stack_example}

			\end{figure}

			\subsubsection*{Value Stack}
			\label{subsubsec:value_stack}
			\par This stack is also known as the evaluation stack. It is where the manipulation of the object happens when evaluating object-manipulating opcodes.
			A value stack is found in a call-frame, implying bijectivity. Any manipulations performed on this stack (unless they are namespace related) are independent of other stacks and
			do not have the permissions to push values on other value stacks.
			
			\subsubsection*{Block Stack}
			\label{subsubsec:block_stack}

			\par The block stack keeps track of different types of control structures, such as; loops, try/except blocks, and with blocks. These structures push entries onto the block stack, which are popped whenever exiting
			the said structure. The block stack allows the interpreter to keep track of active blocks at any moment

			\begin{figure}
				\centering

				\begin{tikzpicture}
					[
						roundnode/.style={circle, draw=green!60, fill=green!5, very thick, minimum size=20mm},
						rectanglenode/.style={rectangle, draw=cyan!60, fill=cyan!20, very thick, minimum width=30mm, minimum height = 10mm},
						transparentbox/.style={rectangle,draw=blue!60, fill=blue!0, ultra thick, minimum width = 75mm, minimum height = 40mm},
					]

					\node[rectanglenode] (callstack) at (0,0) {Call Stack};

					\node[rectanglenode] (framestack) [right=of callstack] {Frame Stack};
					\node[rectanglenode] (blockstack) [above=of framestack] at (8,0) {Block Stack};
					\node[rectanglenode] (valuestack) [below=of framestack] at (8,0){Value Stack};
					

					\draw[-latex] (callstack) -- (framestack);
					\draw[-latex] (framestack.east) -- ++(0.2,0)|-  (blockstack.west);
					\draw[-latex] (framestack.east) -- ++(0.2,0)|-  (valuestack.west);

				\end{tikzpicture}

				\caption{Overview of CPython stacks}
				\label{fig:stacks_overview}
			\end{figure}

		\pagebreak

		\subsection{Frames}
		\label{subsec:frames}
		\par A frame (call-frame) is an object which represents a current function call (subprogram call); more formally referred to as a code object. It is an internal type containing administrative information useful for debugging and is used
		by the interpreter \cite[pp.18--19]{van1994python}. Frame objects are tightly coupled with the three main stacks (See {\bfseries\nameref{subsec:stacks}}) by which every frame is linked to another. Every frame object has two frame-specific stacks;
		value stack (See {\bfseries\nameref{subsubsec:value_stack}}) and the block stack (See {\bfseries\nameref{subsubsec:block_stack}}). Frames are born from function calls, and die when that function is returned.
		
			\subsubsection*{Frame Attributes}
			\par Along with the properties mentioned above, a frame object would have the following attributes:
			\begin{description}
				\item [f\_back] points to the previous stack frame object (return address).
				\item [f\_code] points to the current code object (See {\bfseries\nameref{subsec:code_obj}}) being executed, in the current frame.
				\item [f\_builtin] points to the builtin symbol table.
				\item [f\_globals] points to the dictionary used to look up global variables.
				\item [f\_locals] points to the symbol table used to look up local variables.
				%\item [f\_valuestack] this is a pointer, which points to the address after the last local variable. TODO:WRONG
				\item [f\_valuestack] this holds the pointer, pointing to the value of the top of the value stack.
				\item [f\_lineno] this gives the line number of the frame.
				\item [f\_lasti] this gives the bytecode instruction offset of the last instruction called.
				\item [f\_blockstack] contains the block state, and block relations.
				\item [f\_localsplus] is a dynamic structure that holds any values in the value stack, for evaluation purposes.
			\end{description}

			These attributes are retrieved from the declaration found in \textit{./Include/frameobject.h}.

			\subsubsection*{Frame Stack}
			\par The stack frame is a collection of all the current frames in a call-stack-like data structure (See {\bfseries\nameref{subsubsec:call_stack}}).
			A frame is pushed onto the stack frame for every function call (every function has a unique frame) as shown in Figure \ref{fig:stack_frames}.
			\par The stack frame contains a frame pointer which is another register that is set to the current stack frame. Frame pointers resolve the issue created 
			when operations (pushes or pops) are computed on the stack hence changing the stack pointer, invalidating any hard-coded offset addresses that are computed statically, before run-time \cite[]{stack2011csuwm}.
			With frame pointers, references to the local variables are offsets from the frame pointer, not the stack pointer.
			\begin{figure}[H]
				
				\centering
				\begin{tikzpicture}
					[
						roundnode/.style={circle, draw=green!60, fill=green!5, very thick, minimum size=20mm},
						rectanglenode_main/.style={rectangle, draw=cyan!60, fill=cyan!20, very thick, minimum width=30mm, minimum height = 10mm},
						rectanglenode_checkFile/.style={rectangle, draw=yellow!60, fill=yellow!20, very thick, minimum width=30mm, minimum height = 10mm},
						rectanglenode_openFile/.style={rectangle, draw=red!60, fill=red!20, very thick, minimum width=30mm, minimum height = 10mm},
						rectanglenode_addition/.style={rectangle, draw=orange!60, fill=orange!20, very thick, minimum width=30mm, minimum height = 10mm},
						transparentbox/.style={rectangle,draw=blue!60, fill=blue!0, ultra thick, minimum width = 75mm, minimum height = 40mm},
					]
					
					
					\node[rectanglenode_main] (mainfunc) at (0,0) {Main};
					\node[rectanglenode_checkFile] (func1) [below=of mainfunc] {checkFile};
					\node[rectanglenode_openFile] (func2) [below=of func1] {openFile};
					\node[rectanglenode_addition] (func3) [below=of func2] {additionNumbs};
					
					\node[rectanglenode_main] (mainfunc_frame1) [right= of mainfunc] at (5,0) {Return Address};
					\node[rectanglenode_addition] (addition_frame1) [right= of mainfunc] at (5,-1) {Local Variable(z)};
					\node[rectanglenode_addition] (addition_frame1) [right= of mainfunc] at (5,-2) {Local Variable(y)};
					\node[rectanglenode_addition] (addition_frame1) [right= of mainfunc] at (5,-3) {Local Variable(x)};
					\node[rectanglenode_addition] (addition_frame1) [right= of mainfunc] at (5,-4) {Return Address};
					\node[rectanglenode_openFile] (openFile_frame1) [right= of func1] at(5,-5){Local Variable(filename)};
					\node[rectanglenode_openFile] (openFile_frame2) [right= of func1] at(5,-6){Parameter("addition\_variables.txt")};
					\node[rectanglenode_openFile] (openFile_frame3) [right= of func1] at(5,-7){Return Address};
					\node[rectanglenode_checkFile] (checkFile_frame1) [right= of func1] at(5,-8){Local Variable(filepath)};
					\node[rectanglenode_checkFile] (checkFile_frame2) [right= of func1] at(5,-9){Parameter(filenamecheck)};
					
					\node[] (textfunctions) [above=of mainfunc] {\bfseries{Functions}};
					\node[] (textstack) [above=of mainfunc_frame1] {\bfseries{Stack Frame}};

				\end{tikzpicture}
				\caption{Stack frames example}
				\label{fig:stack_frames}
			
			\end{figure}

		\subsection{Code Objects}
		\label{subsec:code_obj}
		
		\par A code object is a low-level detail of the CPython implementation. When parsing Python code, compilation creates a code object for processing on the \acs{PVM}. A code object contains a list of instructions directly interacting with the CPython \acs{VM}; hence coined a low-level detail. Code objects are of the type \lstinline|PyCodeObject|, with each section of the code object representing 
		a chunk of executable code that has not been bound to a function \cite[]{pythonofficial2022docspycode}. The structure of the type of these code 
		objects change throughout different CPython versions, thus there is no set composition \cite[]{pythonofficial2022docspycode}. For reference, the source code in Listing \ref{lst:dis_example} produces
		the code object displayed in Listing \ref{lst:codeobjscript}, following the standard convention shown in Listing \ref{lst:codeobjconv}.
		
		\begin{lstlisting}[float=h,language=bash, caption= Code object of Listing \ref{lst:dis_example},numbers=none]
			<code object addition_numbers at 0x1047f10b0, file "filepath", line 3>
		\end{lstlisting}\label{lst:codeobjscript}
		
		\small
		\begin{lstlisting}[float=h,language=bash, caption= Code object standard convention,numbers=none]
			code object <functionName> at <address>, file <path>, line<firstLineNo.>
		\end{lstlisting}\label{lst:codeobjconv}
		
		\normalsize

			\subsubsection*{Disassembler}
			\par Code objects are expanded by using the \lstinline|dis| module in Python. This module contains several analyses functions which; all of which directly convert the input code object into the desired output \cite[]{pythonofficial2022docsdismodule}.
			The function that is of a particular interest in this paper is the \lstinline|dis.dis| function which disassembles a code object into its respective bytecodes, alongside other relevant information, as seen in Figure \ref*{lst:dis_example}.
			When applying the analysis function \lstinline|dis.dis|, the disassembled code object takes the following format for every instruction:
			
			\small
			\begin{lstlisting}[language=bash, caption= Dissasembled instruction convention,numbers=none]
				<lineNumber><label><instructionOffset><opname><opargs><var>
			\end{lstlisting}
			\normalsize

			It is interesting to note that the value for \lstinline|opargs| is computed in little-endian order. Typically, as is shown in ... ,the arguments associated with the instructions are used for 
			specific stack manipulations. 
			%%TODO: Maybe add some more detail
			%%TODO: Add the correct reference section above.

				
				\begin{lstlisting}[language=Python,caption=Python source code,label=lst:addnossource]
				        import dis
				        
				        def addition_numbers(x,y):
				            z=x+y
				            return z
				       
				        dis.dis(addition_numbers)
				\end{lstlisting}

				\begin{lstlisting}[numbers=none, caption=Disassembly of Listing \ref{lst:addnossource}]
				4    	0 LOAD_FAST        0(x)
							2 LOAD_FAST        1(y)
							4 BINARY_ADD
							6 STORE_FAST       2(z)
				
				5		  LOAD_FAST        2(z)
							10 RETURN_VALUE
				\end{lstlisting}\label{lst:dis_example}
%			\end{figure}

			\subsubsection*{Bytecode}
			\label{subsubsec:bytecode}
			
			\par Bytecode is a form of portable code (p-code) executed on a virtual machine. 
			The introduction of the concept of bytecode came about when a generalized way 
			 of interpreting complex programming languages was required to simplify information structures, characterizing them in their essentials \cite[]{landin1964mechanical}.
			 This ideology gave birth to portable code; a generalized form of code, that can cross-compile, assuming that
			 the Virtual Machine which interpreted the p-code was compatible with the native machine architecture. 
			 \par In CPython V3.10, there are over 100 low-level bytecode instructions \cite[]{pythonofficial2022docsdismodule}. The classification of bytecode operations is defined below:
			\begin{itemize}
				\item[-]Unary instructions.
				\item[-] Binary instructions.
				\item[-] Inplace instructions.
				\item[-] Coroutine instructions.
				\item[-] Argument instructions.
				\item[-] Miscellaneous instructions.
			\end{itemize}
			
		\subsection{Execution of Code Objects}
		The evaluation stage firstly makes use of the public API \lstinline|PyEval_EvalCode()| \cite[lines 716--724]{ceval2022github}, which is used for evaluating a code object created 
		at the end of the compilation stages (Figure \ref{fig:compiler_innards}). This API constructs an execution frame from the top of the stack by calling \lstinline|_PyEval_EvalCodeWithName()|.
		The first execution frame constructed must conform to these requirements:
		\begin{enumerate}
			\item The resolution of keyword\footnote{A keyword argument is a value that, when passed into a function, is identifiable by a specific parameter name, such as variable assignment.} 
			and positional arguments\footnote{A positional argument is a value that is passed into a function based on the order in which the parameters were listed during the function definition.}.
			\item The resolution of *args\footref{footnote:kwargs_args} and **kwargs\footnote{\label{footnote:kwargs_args}*args and **kwargs allow multiple arguments to be passed into a function via the unpacking (*) operator.} 
			in function definitions.
			\item The addition of arguments as local variables to the scope (A scope is the membership of a variable to a region).
			\item The creation of Co-routines and Generators \cite[pp.2--3]{tismer2000continuations}.
		\end{enumerate} 
		Code execution in CPython is the evaluation and interpretation of code object. Below, we delve into a more detailed description of frame objects; their creation, and execution \cite[]{real2022python}.
		
			\subsubsection*{Thread State Construction}
			\par Prior to execution, the frame would need to be referenced from a thread. The interpreter allows for many threads to run at any given moment. The thread structure that is created is called 
			\lstinline|PyThreadState|.
			
			\subsubsection*{Frame Construction}
			\par Upon constructing the frame, the following arguments are required:
			\begin{description}
				\item [\_co] A \lstinline|PyCodeObject| (code object).
				\item [globals] A dictionary relating global variable names with their values.
				\item [locals] A dictionary relating local variable names with their values.
			\end{description}
			It is important to note that there are other arguments that might be used but do not form part of the basic API, thus will not be included.

			\subsubsection*{Keyword \& Positional Argument Handling}
			\par If the function definition contains a multi-argument keyword argument \footref{footnote:kwargs_args}, a new keyword argument dictionary (\lstinline|kwdict| dictionary) is created in the form of a \lstinline|PyDictObject|.
			Similarly, if any positional arguments\footref{footnote:kwargs_args} are found they are set as local variables.

			\par The dictionary that was created is now filled with the remaining keyword arguments which do not resolve themselves to positional arguments. This resolution comes after
			all the other arguments have been unpacked. In addition, missing positional arguments \footnote{Positional arguments that are provided to a function call, but are not in the list of positional arguments.} are 
			added to the \lstinline|*args|\footref{footnote:kwargs_args} tuple. The same process is followed for the keyword arguments; values are added to the \lstinline|**kwargs| dictionary and not a tuple.

			\subsubsection*{Final Stage}
			\par Any closure names are added to the code object's list of free variable names, and finally, generators and coroutines are handled in a new frame. In this case, the frame is not pre-evaluated but it is 
			evaluated only when the generator/coroutine method is called to execute its target.


		\subsection{Execution of Frame Objects}
		\par The local and global variables are added to the frame preceding frame evaluation; handled by \lstinline|_PyEval_EvalFrameDefault()|. 
		\par\lstinline|_PyEval_EvalFrameDefault()| is the central function which is found in the main execution loop. Anything that CPython executes goes through
		this function and forms a vital part of interpretation.
